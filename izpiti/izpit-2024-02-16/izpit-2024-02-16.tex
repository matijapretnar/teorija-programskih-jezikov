\documentclass[arhiv]{../izpit}
\usepackage{amssymb}
\usepackage{fouriernc}
\usepackage{mathpartir}
\usepackage{stmaryrd}

\begin{document}

\newcommand{\bnfis}{\mathrel{{:}{:}{=}}}
\newcommand{\bnfor}{\;\mid\;}
\newcommand{\fun}[2]{\lambda #1. #2}
\newcommand{\minfix}[2]{\mu #1. #2}
\newcommand{\conditional}[3]{\mathtt{if}\;#1\;\mathtt{then}\;#2\;\mathtt{else}\;#3}
\newcommand{\whileloop}[2]{\mathtt{while}\;#1\;\mathtt{do}\;#2}
\newcommand{\recfun}[3]{\mathtt{rec}\;#1\;#2. #3}
\newcommand{\boolty}{\mathtt{bool}}
\newcommand{\intty}{\mathtt{int}}
\newcommand{\funty}[2]{#1 \to #2}
\newcommand{\tru}{\mathtt{true}}
\newcommand{\fls}{\mathtt{false}}
\newcommand{\unt}{\hbox{\texttt{()}}}
\newcommand{\tbool}{\mathtt{bool}}
\newcommand{\tand}{\mathbin{\mathtt{and}}}
\newcommand{\tandalso}{\mathbin{\mathtt{andalso}}}
\newcommand{\imp}{\textsc{imp}}
\newcommand{\skp}{\mathtt{skip}}
\newcommand{\itp}[1]{\llbracket #1 \rrbracket}
\makeatletter
\newcommand{\nadaljevanje}{\dodatek{\newpage\noindent\emph{(\@sloeng{nadaljevanje rešitve \arabic{naloga}. naloge}{continuation of the answer to question \arabic{naloga}})}}}
\makeatother
\izpit
[ucilnica=203,naloge=-1]{Teorija programskih jezikov: 2. izpit}{16.\ februar 2024}{
}
\dodatek{
  \vspace{\stretch{1}}
  \begin{itemize}
    \item \textbf{Ne odpirajte} te pole, dokler ne dobite dovoljenja.
    \item Zgoraj \textbf{vpišite svoje podatke} in označite \textbf{sedež}.
    \item Na vidno mesto položite \textbf{dokument s sliko}.
    \item Preverite, da imate \textbf{telefon izklopljen} in spravljen.
    \item Čas pisanja je \textbf{180 minut}.
    \item Doseči je možno \textbf{80 točk}.
    \item Veliko uspeha!
  \end{itemize}
  \vspace{\stretch{3}}
  \newpage
}

%%%%%%%%%%%%%%%%%%%%%%%%%%%%%%%%%%%%%%%%%%%%%%%%%%%%%%%%%%%%%%%%%%%%%%%

\naloga[\tocke{20}]

V $\lambda$-računu predstavimo kombinatorje S, K in I z izrazi:
\begin{align*}
  \mathsf{s} & = \fun{x} \fun{y} \fun{z} (x \, z) \, (y \, z) \\
  \mathsf{k} & = \fun{x} \fun{y} x                            \\
  \mathsf{i} & = \mathsf{s} \, \mathsf{k} \, \mathsf{k}
\end{align*}

\podnaloga[\tocke{5}] Naj bo $V$ poljubna vrednost. Zapišite vse korake v neučakani evalvaciji $\mathsf{i} \, V \leadsto^* V$. Izpeljav posameznih korakov ni treba pisati.
\podnaloga[\tocke{10}] S pomočjo Hindley-Milnerjevega algoritma izračunajte najbolj splošen tip izraza $\mathsf{s}$.
\podnaloga[\tocke{5}] Dokažite, da izraz $(\mathsf{s} \, \mathsf{i} \, \mathsf{i}) \, (\mathsf{s} \, \mathsf{i} \, \mathsf{i})$ nima veljavnega tipa.

\nadaljevanje

%%%%%%%%%%%%%%%%%%%%%%%%%%%%%%%%%%%%%%%%%%%%%%%%%%%%%%%%%%%%%%%%%%%%%%%

\naloga[\tocke{20}]
\newcommand{\nxt}{\mathtt{next}}
V $\lambda$-račun z neučakano semantiko dodamo izraz $\nxt$, ki ob vsakem klicu vrne naslednji člen iz vnaprej danega neskončnega zaporedja celih števil:
\[
  \text{izraz $e$} \bnfis
  x \bnfor
  \underline{n} \bnfor
  \tru \bnfor
  \fls \bnfor
  e_1 + e_2 \bnfor
  e_1 = e_2 \bnfor
  \conditional{e}{e_1}{e_2} \bnfor
  \fun{x}{e} \bnfor
  e_1 \; e_2 \bnfor
  \nxt
\]
Na primer, izraz $\nxt + \nxt = \nxt$ izračuna, če je vsota prvih dveh členov enaka tretjemu členu.

Za razširjeni $\lambda$-račun zapišite pravila za semantiko malih korakov oblike $\langle e \mid s\rangle \leadsto \langle e' \mid s'\rangle$, da se bo zgornji primer ob zaporedju praštevil $2, 3, 5, 7, \dots$ evalviral kot
\begin{align*}
  \langle \nxt + \nxt = \nxt & \mid 2, 3, 5, \dots \rangle \leadsto   \\
  \langle 2 + \nxt = \nxt    & \mid 3, 5, 7, \dots \rangle \leadsto   \\
  \langle 2 + 3 = \nxt       & \mid 5, 7, 11, \dots \rangle \leadsto  \\
  \langle 5 = \nxt           & \mid 5, 7, 11, \dots \rangle \leadsto  \\
  \langle 5 = 5              & \mid 7, 11, 13, \dots \rangle \leadsto \\
  \langle \tru               & \mid 7, 11, 13, \dots \rangle
\end{align*}
\nadaljevanje

%%%%%%%%%%%%%%%%%%%%%%%%%%%%%%%%%%%%%%%%%%%%%%%%%%%%%%%%%%%%%%%%%%%%%%%

\naloga[\tocke{20}]
\newcommand{\fix}{\mathsf{fix}}


\podnaloga[\tocke{5}]
Naj bo $D$ domena in $f : D \to D$ zvezna preslikava.
Pokažite, da za poljuben $y \in D$ velja
\[
  y \ge f(y) \implies y \ge \minfix{x}{f(x)}
\]
kjer je $\minfix{x}{f(x)}$ alternativen zapis za $\fix(f)$, torej najmanjšo fiksno točko preslikave $f$.

\podnaloga[\tocke{15}]
Dani naj bosta domeni $D$, $E$ ter zvezni preslikavi $f \colon D \times E \to D$ in $g \colon D \times E \to E$. Definirajmo
%
\begin{align*}
  x_0 & = \minfix{x}{f(x, \minfix{y}{g(x, y)})} \\
  y_0 & = \minfix{y}{g(x_0, y)}
\end{align*}
%
Predpostavite lahko, da so vse fiksne točke dobro definirane, torej da našteti predpisi res podajajo zvezne preslikave.

Pokažite, da sta $x_0$ in $y_0$ najmanjši hkratni fiksni točki $f$ in $g$, torej najmanjši vrednosti, da velja $x_0 = f(x_0, y_0)$ in $y_0 = g(x_0, y_0)$.

\nadaljevanje

%%%%%%%%%%%%%%%%%%%%%%%%%%%%%%%%%%%%%%%%%%%%%%%%%%%%%%%%%%%%%%%%%%%%%%%

\naloga[\tocke{20}]
\newcommand{\skipcmd}{\mathtt{skip}}

Programi v jeziku \textsc{MiniWasm} so sestavljeni iz zaporedij ukazov, ki jih podamo s sintakso:
\begin{align*}
  \text{ukaz $i$} \bnfis {}
   & \mathtt{push}_v \bnfor
   \mathtt{drop} \bnfor
  \mathtt{add} \bnfor
  \mathtt{eq} \bnfor
  \mathtt{if} \, is_1 \, is_2 \\
  \text{program $is$} \bnfis {}
   & [] \bnfor
  i :: is                     \\
  \text{vrednost $v$} \bnfis {}
   & \underline{n} \bnfor
  \tru \bnfor
  \fls
\end{align*}
%
\textsc{MinWasm} je skladovni jezik, kar pomeni, da programi ob izvajanju spreminjajo sklad vrednosti $vs = v_1 :: v_2 :: \cdots :: v_k :: []$.
Operacijsko semantiko definiramo z relacijo $is \mid vs \hookrightarrow is' \mid vs'$, podano kot
%
\begin{mathpar}
  \infer{
  }{
    (\mathtt{push}_v :: is) \mid vs \hookrightarrow is \mid (v :: vs)
  } \and
  \infer{
  }{
    (\mathtt{drop} :: is) \mid (v :: vs) \hookrightarrow is \mid vs
  } \and
  \infer{
  }{
    (\mathtt{add} :: is) \mid (\underline{n_1} :: \underline{n_2} :: vs) \hookrightarrow is \mid (\underline{n_1 + n_2} :: vs)
  } \and
  \infer{
    n_1 = n_2
  }{
    (\mathtt{eq} :: is) \mid (\underline{n_1} :: \underline{n_2} :: vs) \hookrightarrow is \mid (\tru :: vs)
  } \and
  \infer{
    n_1 \ne n_2
  }{
    (\mathtt{eq} :: is) \mid (\underline{n_1} :: \underline{n_2} :: vs) \hookrightarrow is \mid (\fls :: vs)
  } \and
  \infer{
  }{
    ((\mathtt{if} \, is_1 \, is_2) :: is) \mid (\tru :: vs) \hookrightarrow (is_1 @ is) \mid vs
  } \and
  \infer{
  }{
    ((\mathtt{if} \, is_1 \, is_2) :: is) \mid (\fls :: vs) \hookrightarrow (is_2 @ is) \mid vs
  }
\end{mathpar}
%
kjer je $@$ stikanje seznamov.

Če vrednostim in skladom na očiten način z relacijama $\vdash v : t$ in $\vdash vs : ts$ priredimo tipe
%
\begin{align*}
  \text{tip vrednosti $t$} \bnfis {}
   & \mathtt{int} \bnfor
  \mathtt{bool}          \\
  \text{tip sklada $ts$} \bnfis {}
   & [] \bnfor
  t :: ts
\end{align*}
%
lahko definiramo tromestni relaciji $\vdash i : ts_1 \to ts_2$ in $\vdash is : ts_1 \to ts_2$, ki povesta, da ukaz~$i$ oziroma program~$is$ sklad tipa $ts_1$ spremenita v sklad tipa $ts_2$:
%
\begin{mathpar}
  \infer{
    \vdash v : t
  }{
    \vdash \mathtt{push}_v : ts \to (t :: ts)
  } \and
  \infer{
  }{
    \vdash \mathtt{drop} : (t :: ts) \to ts
  } \and
  \infer{
  }{
    \vdash \mathtt{add} : (\mathtt{int} :: \mathtt{int} :: ts) \to (\mathtt{int} :: ts)
  } \and
  \infer{
  }{
    \vdash \mathtt{eq} : (\mathtt{int} :: \mathtt{int} :: ts) \to (\mathtt{bool} :: ts)
  } \and
  \infer{
    \vdash is_1 : ts_1 \to ts_2 \and
    \vdash is_2 : ts_1 \to ts_2
  }{
    \vdash (\mathtt{if} \, is_1 \, is_2) : (\mathtt{bool} :: ts_1) \to ts_2
  } \\
  \infer{
  }{
    \vdash [] : ts \to ts
  } \and
  \infer{
    \vdash i : ts_1 \to ts_2 \and
    \vdash is : ts_2 \to ts_3
  }{
    \vdash (i :: is) : ts_1 \to ts_3
  }
\end{mathpar}
%
Zapišite trditvi o ohranitvi in napredku ter eno od njiju dokažite.

\nadaljevanje

\end{document}